\documentclass[12pt,twoside]{report}
%%%%%%%%%%%%%%%%%%%%%%%
\usepackage{orcidlink}
%\usepackage{lmodern}
\usepackage{graphicx}
\graphicspath{{./figures/figures_MSSM/}{./figures/figures_BLSSM/}{./figures/figures_LRIS/}{./figures/figures_VLSM/}}
\usepackage{adjustbox}
\usepackage{booktabs}
\usepackage{xcolor}
%\usepackage[table,xcdraw]{xcolor}
\usepackage{multirow}
\usepackage{rotating,tabularx}
\usepackage[intoc]{nomencl}
\renewcommand{\nomname}{List of Abbreviations}
\makenomenclature
\usepackage{pifont}
\usepackage{slashbox}
\newcommand{\cmark}{\ding{51}}%
\newcommand{\xmark}{\ding{55}}%
\usepackage{chngcntr}
\usepackage{axodraw}
\usepackage[centertags]{amsmath}
\usepackage{amsfonts}
\usepackage{amssymb}
\usepackage{amsthm}
\usepackage{newlfont}
\usepackage{xthesis}
\usepackage{xtocinc}
\usepackage{float}
\usepackage{caption}
\usepackage{subcaption}
\usepackage[avantgarde]{quotchap}
\usepackage{fancyhdr}
\usepackage[bookmarks=true,bookmarksnumbered=false,bookmarksopen=true,colorlinks=true,linkcolor=webred,breaklinks=true,citecolor=blue]{hyperref}
\usepackage[hyphenbreaks]{breakurl}
\usepackage{url}
\definecolor{webgreen}{rgb}{0, 0.5, 0}
\definecolor{webblue}{rgb}{0, 0, 0.5}
\definecolor{webred}{rgb}{0.5, 0, 0}
\usepackage{slashed}
\usepackage{colortbl}
\usepackage{array}
\usepackage{color}
\usepackage{arydshln}
\usepackage{lscape}
\usepackage{changemargin}
\usepackage{cite,amssymb}
\usepackage{amsmath} 
\usepackage{graphics}
\usepackage{graphicx}
\usepackage{slashbox}
\usepackage{hyperref}
\usepackage[normalem]{ulem}
\usepackage[compat=1.0.0]{tikz-feynman}
\renewcommand*{\theenumi}{\thesection.\arabic{enumi}}
\renewcommand*{\theenumii}{\theenumi.\arabic{enumii}}
\usepackage{pgfplots}
%%%%%%%%%%%%%%%%%%%%%%%%%%%%%%%%%%%%%%%%%%%%%%%%%%%%%%%%%%%%%%%%%%%%%%%%%
\def\lsim{\raise0.3ex\hbox{$\;<$\kern-0.75em\raise-1.1ex\hbox{$\sim\;$}}}
\def\gsim{\raise0.3ex\hbox{$\;>$\kern-0.75em\raise-1.1ex\hbox{$\sim\;$}}}
\def\ie{{\it i.e.,~}}
%%%%%%%%%%%%%%%%%%%%%%%%
\def\ththMat#1#2#3#4#5#6#7#8#9{\ensuremath{\begin{pmatrix}#1&#2&#3\\#4&#5&#6\\#7&#8&#9\end{pmatrix}}}
\def\twtwMat#1#2#3#4{\ensuremath{\begin{pmatrix}#1&#2\\#3&#4\\\end{pmatrix}}}
\def\thtwMat#1#2#3#4#5#6{\ensuremath{\begin{pmatrix}#1&#2\\#3&#4\\#5&#6\end{pmatrix}}}
\def\fivecMat#1#2#3#4#5{\ensuremath{\begin{pmatrix}#1\\#2\\#3\\#4\\#5\end{pmatrix}}}
\def\threecMat#1#2#3{\ensuremath{\begin{pmatrix}#1\\#2\\#3\end{pmatrix}}}
\def\fiverMat#1#2#3#4#5{\ensuremath{\begin{pmatrix}#1&#2&#3&#4&#5\end{pmatrix}}}
\def\threerMat#1#2#3{\ensuremath{\begin{pmatrix}#1&#2&#3\end{pmatrix}}}
\def\tworMat#1#2{\ensuremath{\begin{pmatrix}#1&#2\end{pmatrix}}}
\def\twocMat#1#2{\ensuremath{\begin{pmatrix}#1\\#2\end{pmatrix}}}
\def\Matdiag#1#2#3{\ensuremath{\begin{pmatrix}#1&0&0\\0&#2&0\\0&0&#3\end{pmatrix}}}
\def\MatCase#1#2{\ensuremath{\begin{matrix}#1\\#2\end{matrix}}}
\def\MatU{\ensuremath{\frac{1}{\sqrt{3}}\begin{pmatrix}1 &1 &1\\1 &\omega &\omega^2\\1 &\omega^2 &\omega\end{pmatrix}}}
\newcommand{\da}[1]{16\pi^2\;\frac{d #1}{dt}}
\newcommand{\mat}[1]{\begin{pmatrix} #1 \end{pmatrix}}
\newcommand{\UBL}{{\ensuremath{U(1)_{B-L}}}\xspace}
\newcommand{\higgsbounds}{{\sc HiggsBounds}}
\newcommand{\higgssignals}{{\sc HiggsSignals}}
\makeatletter
\renewcommand{\baselinestretch}{1.23}
\renewcommand\chapterheadstartvskip{\vspace*{-\baselineskip}}
\newtheorem{theorem}{Theorem}
\newtheorem{corollary}{Corollary}
%%%%%%%%%%%%%%%%%%%%%%%
\hfuzz2pt
%%%%%%%%%%%%%%%%%%%%%%%
\newlength{\defbaselineskip}
\setlength{\defbaselineskip}{\baselineskip}
\newcommand{\setlinespacing}[1]{\setlength{\baselineskip}{#1 \defbaselineskip}}
\newcommand{\doublespacing}{\setlength{\baselineskip}{2.0 \defbaselineskip}}
\newcommand{\singlespacing}{\setlength{\baselineskip}{\defbaselineskip}}
%%%%%%%%%%%%%%%%%%%%%%%
\newcommand{\SMsym}{$SU(3)_{_{C}}\times SU(2)_{_{L}}\times U(1)_{_{Y}}$}
\newcommand{\LRMsym}{SU(3)_{_{C}}\times SU(2)_{_{L}}\times SU(2)_{_{R}}\times U(1)_{_{B-L}}}
\newcommand{\Lg}{{\cal L}}
\newcommand{\1}{\left}
\newcommand{\2}{\right}
\newcommand{\g}{\gamma}
\newcommand{\p}{\partial}
\newcommand{\lb}{\label}
\newcommand{\rf}[1]{(\ref{#1})}
\newcommand{\eq}[1]{eq.~(\ref{#1})}
\newcommand{\be}{\begin{equation}\displaystyle}
\newcommand{\ee}{\end{equation}}
\newcommand{\bea}{\begin{eqnarray}\displaystyle}
\newcommand{\eea}{\end{eqnarray}}
\newcommand{\bes}{\begin{equation*}\displaystyle}
\newcommand{\ees}{\end{equation*}}
\newcommand{\beas}{\begin{eqnarray*}\displaystyle}
\newcommand{\eeas}{\end{eqnarray*}}
\newcommand{\n}{\nonumber\\}
\newcommand{\s}{\smallskip}
\newcommand{\tb}{\tan \beta}
\newcommand{\non}{\nonumber}
\newcommand{\ra}{\rightarrow}
\newcommand{\dd}{\displaystyle}
\newcommand{\pd}{\partial}
\newcommand{\ovl}{\overline}
\newcommand{\lm}{\lambda}
\newcommand{\tw}{\theta_{_{W}}}
\newcommand{\al}{\alpha}
\newcommand{\red}[1]{\textcolor[rgb]{1,0,0}{#1}}
\newcommand{\green}[1]{\textcolor[rgb]{0,1,0}{#1}}
\newcommand{\blue}[1]{\textcolor[rgb]{0,0,1}{#1}}
\newcommand{\feynrules}{{\sc FeynRules}}
\newcommand{\FRversion}{2.0}
\newcommand{\asperge}{{\sc ASperGe}}
\newcommand{\gsl}{{\sc Gsl}}
\newcommand{\feynarts}{{\sc FeynArts}}
\newcommand{\feyncalc}{{\sc FeynCalc}}
\newcommand{\mathematica}{{\sc Mathematica\textsuperscript{\textregistered}}}
\newcommand{\calchep}{{\sc CalcHep}}
\newcommand{\comphep}{{\sc CompHep}}
\newcommand{\mgme}{{\sc MadGraph/MadEvent}}
\newcommand{\ufo}{{\sc UFO}}
\newcommand{\sherpa}{{\sc Sherpa}}
\newcommand{\whizard}{{\sc Whizard}}
\newcommand{\formcalc}{{\sc Form\-Calc}}
\newcommand{\gosam}{{\sc GoSam}}
\newcommand{\python}{{\sc Python}}
\newcommand{\madgraph}{{\sc MadGraph}}
\newcommand{\madevent}{{\sc MadEvent}}
\newcommand{\madanalysis}{{\sc MadAnalysis}}
\newcommand{\aloha}{{\sc Aloha}}
\newcommand{\herwig}{{\sc Herwig}}
\newcommand{\lanhep}{{\sc LanHep}}
\newcommand{\ohmega}{{\sc Omega}}
\newcommand{\sarah}{{\sc Sarah}}
\newcommand{\spheno}{{\sc SPheno}}
\newcommand{\cpp}{{\sc C++}}
\newcommand{\pythia}{{\sc Pythia}}
\newcommand{\delphes}{{\sc Delphes}}
\newcommand{\giant}{{\sc Giant}}
\newcommand{\montecarlo}{{\sc MonteCarlo}}
\newcommand{\madwidth}{{\sc MadWidth}}
\newcommand{\vevacious}{{\sc vevacious}}
\newcommand{\tmva}{{\sc TMVA}}
\newcommand{\latex}{\LaTeX\xspace}
\def\lsim{\raise0.3ex\hbox{$\;<$\kern-0.75em\raise-1.1ex\hbox{$\sim\;$}}}
\def\gsim{\raise0.3ex\hbox{$\;>$\kern-0.75em\raise-1.1ex\hbox{$\sim\;$}}}
\def\ie{{\it i.e.}}
\def\bt{\begin{table}}
\def\et{\end{table}}
%%%%%%%%%%
%% VLSM %%
%%%%%%%%%%
%\newcommand{\vev}[1]{{\langle{#1}\rangle}}
\newcommand{\MEW}{m_Z}
\newcommand{\wid}{{\Gamma}} 
\newcommand{\br}[2]{\text{Br}({#1}\to{#2})} 
\newcommand{\ev}[1]{\langle{#1}\rangle} 
\newcommand{\qqad}{\quad\quad\ }
\newcommand{\brbsg}{\text{Br}(b \to s \gamma)}
\newcommand{\brbsm}{\text{Br}(B_s \to \mu^+\mu^-)}
\newcommand{\ep}{\epsilon}
\newcommand{\Ms}{{M_\text{susy}}}
\newcommand{\half}{\frac{1}{2}}
\newcommand{\lam}{\lambda}
\newcommand{\kap}{\kappa}
\newcommand{\eps}{\epsilon} 
\newcommand{\sig}{\sigma}
\newcommand{\bchi}{\bar{\chi}}
\newcommand{\blam}{\bar{\lambda}}
\newcommand{\bpsi}{\bar{\psi}}
\newcommand{\bPsi}{\bar{\Psi}}
\newcommand{\tchi}{\tilde{\chi}}
\newcommand{\tlam}{\tilde{\lambda}}
\newcommand{\tpsi}{\tilde{\psi}}
\newcommand{\tPsi}{\tilde{\Psi}}
\newcommand{\hchi}{\hat{\chi}}
\newcommand{\hlam}{\hat{\lambda}}
\newcommand{\hpsi}{\hat{\pai}}
\newcommand{\hPsi}{\hat{\Psi}}
\newcommand{\Uop}{\mathrm{\U1'}}
\newcommand{{\lag}}{\mathcal{L}}
\newcommand{\Hcal}{\mathcal{H}}
\newcommand{\Lcal}{\mathcal{L}}
\newcommand{\Ncal}{\mathcal{N}}
\newcommand{\hcal}{\mathcal{H}}
\newcommand{\Ocal}{\mathcal{O}}
\newcommand{\Pcal}{\mathcal{P}}
\newcommand{\Mcal}{\mathcal{M}}
\newcommand{\Ical}{\mathcal{I}}
\newcommand{\Ecal}{\mathcal{E}}
\newcommand{\cred}[1]{{\color{red}#1}}

\setcounter{footnote}{0}
\newcommand{\la}{{\lambda}}
\newcommand{\ka}{{\kappa}}
\newcommand{\mQ}{{m^2_{\tilde{Q}}}}
\newcommand{\mU}{{m^2_{\tilde{u}}}}
\newcommand{\mD}{{m^2_{\tilde{d}}}}
\newcommand{\mL}{{m^2_{\tilde{L}}}}
\newcommand{\mE}{{m^2_{\tilde{e}}}}
\newcommand{\mhu}{{m^2_{H_u}}}
\newcommand{\mhd}{{m^2_{H_d}}}
\newcommand{\ms}{{m^2_S}} 
\newcommand{\Ala}{{A_\lambda}}
\newcommand{\Aka}{{A_\kappa}} 
\newcommand{\order}[1]{\mathcal{O}\left({#1}\right)}
\newcommand{\Zp}{{A'}} 
\newcommand{\pr}{\prime}
\newcommand{\Eps}{\mathcal{E}}
\newcommand{\define}{=}
\newcommand{\GWS}{\mathrm{GWS}}
\newcommand{\abs}[1]{\left|{#1}\right|}
\newcommand{\ol}[1]{\overline{#1}}
\newcommand{\gp}{{g^\prime}}
\newcommand{\MeV}{\mathrm{MeV}}
\newcommand{\GeV}{\mathrm{GeV}}
\newcommand{\TeV}{\mathrm{TeV}}
\newcommand{\rep}[1]{\mathbf{#1}}
\newcommand{\SM}{{\mathrm{SM}}}
\newcommand{\gmt}{g\mathrm{-}2}
%\newcommand{\exp}{{\mathrm{exp}}}
\def\lsim{\raise0.3ex\hbox{$\;<$\kern-0.75em\raise-1.1ex\hbox{$\sim\;$}}}
\def\gsim{\raise0.3ex\hbox{$\;>$\kern-0.75em\raise-1.1ex\hbox{$\sim\;$}}}
\newcommand\Tstrut{\rule{0pt}{2.6ex}} % = `top' strut
\newcommand\Bstrut{\rule[-0.9ex]{0pt}{0pt}} % = `bottom' strut 
%%% COLOR CODE %%%%%%%%%%%%%%%%%%%%%%%%%%%%%
\newcommand{\JK}[1]{\textcolor{red}{#1}}
\newcommand{\WA}[1]{\textcolor{blue}{#1}}
%\newcommand{\AM}[1]{\textcolor{green}{#1}}
%\newcommand{\MA}[1]{\textcolor{magenta}{#1}}
%\newcommand{\tbu}[1]{\textcolor{cyan}{#1}} % 'to be updated' after new result 
%%%%%%%%%%%%%%%%%%%%%%%%%%%%%%%%%%%%%%%%%%%%

%%%%%%%%%%%%%%%%%%%%%%%
\numberwithin{equation}{section}
\renewcommand{\theequation}{\thesection.\arabic{equation}}
\renewcommand{\rm}{\textrm}
\newcommand{\vev}[1]{\left\langle #1 \right\rangle}
%%%%%%%%%%%%%%%%%%%%%%%
% AElsayed
%%%%%%%%%%%%%%%%%%%%%%%
%\newcommand{\1}{\mu_1^2}
%\newcommand{\2}{\mu_2^2}
%\newcommand{\3}{\mu_3^2}
%\newcommand{\4}{\theta}
%\newcommand{\5}{M_{_{Z'}}^2}
%\newcommand{\g}{g''^2}
%\newcommand{\n}{\nonumber\\}
\newcommand{\nn}{\tilde{\nu}}
%%%%%%%%%%%%%%%%%%%%%%%
\setlength{\tclineskip}{1.05\baselineskip}
%%%%%%%%%%%%%%%%%%%%%%%
\dedicate{
\vspace{0.5in}
\texttt{
{\bf To}\\~\\
my wife, Samira
\\\&\\
my little angel, Aisha
\\\&\\
my beloved parents ...
%\\\&\\
%The peaceful souls of my late brothers:
%\\
%Mohamed Ashry (1976-2008) 
%\\and\\ 
%Ahmed Ashry (1983-2004)
%\\\&\\
%The smart souls of my late friends and colleagues:
%\\
%Waleed A. Elsayed (1986-2015)
%\\and\\ 
%Ahmed Elsayed (1986-2013)
%\\\&\\
%Palestinian people who stand as example of freedom
}
}
\phd
\copyrightyear{2024}
\submitdate{October 2024}
\convocation{October}{2024}
%%%%%%%%%%%%%%%%%%%%%%%
\twosupervisors
\title{\textsl{Phenomenological Implications of Non-Minimal Supersymmetric Models}}
\author{Mustafa Ashry Ibrahim Seif}
%
\supervisor{Prof. Ahmad A. Amer\\and\\Prof. Adel A. Mosharafa
\\\small{Department of Mathematics,}
\\\small{Faculty of Science, Cairo University, Egypt}}
%
\firstreader{Prof. Shaaban S. Khalil
\\\small{Center for Fundamental Physics (CFP),}
\\\small{Zewail City of Science and Technology, Egypt}}
%
\examiner{Prof. Qaisar Shafi
\\\small{Department of Physics and Astronomy,}
\\\small{Delaware University, USA}}
%
\secondreader{Prof. Alakabha Datta
\\\small{Department of Physics and Astronomy,}
\\\small{University of Mississippi, USA}}
\university{cairo university}
\dept{Mathematics}
%%%%%%%%%%%%%%%%%%%%%%%

\begin{document}
{Mustafa Ashry\footnote{mustafa@sci.cu.edu.eg,~mustafa@cu.edu.eg,~}{\Huge\orcidlink{0000-0003-4114-3684}}}
\allowdisplaybreaks
{
\typeout{:?000000000} % Don't bother with over/under-full boxes
\beforepreface
\typeout{:?111111111} % Process All Errors from Here on
}
%%%%%%%%%%%%%%%%%%%%%%%
%\setcounter{page}{1}
%%%%%%%%%%%%%%%%%%%%%%%
\printnomenclature[2cm]
{\typeout{Abstract}\include{Preamble/Abstract}}
{\typeout{Acknowledgements}\include{Preamble/Acknowledgment}}
%%%%%%%%%%%%%%%%%%%%%%%
\afterpreface
\def\baselinestretch{1}
%%%%%%%%%%%%%%%%%%%%%%%
\setlinespacing{1.5}
%%%%%%%%%%%%%%%%%%%%%%%
\pagenumbering{arabic} \pagestyle{fancy}
\renewcommand{\chaptermark}[1]{\markboth{\chaptername\ \thechapter:\,\ #1}{}}
\renewcommand{\sectionmark}[1]{\markright{\thesection\,\ #1}}
\addtolength{\headheight}{3pt} \fancyhead{}
\fancyhead[RE]{\sl\leftmark} \fancyhead[RO,LE]{}
\fancyhead[LO]{\sl\rightmark} \fancyfoot[L,E]{}\fancyfoot[C]{\rm\thepage}
%%%%%%%%%%%%%%%%%%%%%%%
{\typeout{Introduction}\include{Preamble/Introduction}}
%%%%%%%%%%%%%%%%%%%%%%%
\include{Chapters/0MSSM}
\include{Chapters/1BLSSM}
\include{Chapters/2BLSSM_Pheno}
\include{Chapters/3LRIS}
\include{Chapters/4LRIS_Pheno}
\include{Chapters/5VLSM}
\include{Preamble/Conclusion}
\include{Preamble/Appendix}
\nomenclature{VLSM}{Vector-Like Fourth Family Extension of the Standard Model}
\nomenclature{QFT}{Quantum Field Theory}
\nomenclature{SM}{Standard Model}
\nomenclature{BSM}{Beyond the Standard Model}
\nomenclature{MSSM}{Minimal Supersymmetric Standard Model}
\nomenclature{BLSSM}{$B-L$ Supersymmetric Standard Model}
\nomenclature{LRSM}{Left-Right Symmetric Model}
\nomenclature{LH}{Left-Handed}
\nomenclature{RH}{Right-Handed}
\nomenclature{EW}{Electroweak}
\nomenclature{VEV}{Vacuum Expectation Value}
\nomenclature{GeV}{Giga Electron Volt}
\nomenclature{TeV}{Tera Electron Volt}
\nomenclature{LHC}{Large Hadron Collider}
\nomenclature{LRIS}{Left-Right Symmetric Model with Inverse Seesaw}
\nomenclature{SSB}{Spontaneous Symmetry Breaking}
\nomenclature{EM}{Electromagnetic}
\nomenclature{LR}{Left-Right}
\nomenclature{MeV}{Mega Electron Volt}
\nomenclature{SUSY}{Supersymmetry}
\nomenclature{$B-L$}{Baryon minus Lepton numbers}
\nomenclature{BP}{Benchmark Point}
\nomenclature{DoF}{Degrees of Freedom}
\nomenclature{LFV}{Lepton Flavor Violation}
\nomenclature{ML}{Machine Learning}
\nomenclature{ROC}{Receiver Operating Characteristic Curve}
\nomenclature{BDT}{Boosted Decision Tree}
\nomenclature{AUC}{Area Under the Curve}
%\nomenclature{EoM}{Equation of Motion}
%\nomenclature{S}{Signal}
%\nomenclature{B}{Background}
%%%%%%%%%%%%%%%%%%%%%%%
\setlinespacing{1.44}
\bibliographystyle{acm1}
\bibliography{./Preamble/References}
%%%%%%%%%%%%%%%%%%%%%%%
\end{document}
%%%%%%%%%%%%%%%%%%%%%%%
For abbreviations list
latex Thesis.tex
bibtex Thesis.aux
latex Thesis.tex
latex Thesis.tex
makeindex Thesis.nlo -s nomencl.ist -o Thesis.nls
latex Thesis.tex
dvips Thesis.dvi
ps2pdf Thesis.ps
%%%%%%%%%%%%%%%%%%%%%%%
For abbreviations list
latex <filename>.tex
makeindex <filename>.nlo -s nomencl.ist -o <filename>.nls
latex <filename>.tex
%%%%%%%%%%%%%%%%%%%%%%%

\iffalse
MEFF 	Effective mass being defined as the sum of the transverse momentum of all final state objects and the missing transverse energy.
MET 	Missing transverse energy.
MHT 	Missing transverse energy defined from the jet activity only.
NPID 	Particle content (PDG code distribution).
NAPID 	Particle content (PDG code distribution in absolute value).
SQRTS 	Partonic center-of-mass energy.
SCALE 	Energy scale of the event.
TET 	Scalar sum of the transverse energy of all final state objects.
THT		Scalar sum of the transverse energy of all final state jets.
WEIGHTS Event weights
ABSETA 	Absolute value of the pseudorapidity.
BETA 	Velocity β = v/c (relatively to the speed of light).
E 		Energy.
EE_HE 	Ratio of the electromagnetic energy to the hadronic energy (for a jet).
ET 		Transverse energy.
ETA 	Pseudorapidity.
GAMMA 	Lorentz-factor.
HE_EE 	Ratio of the hadronic energy to the electromagnetic energy (for a jet).
M 		Invariant mass.
MT 		Transverse mass.
MT_MET 	Transverse mass of the system comprised of the object and the missing momentum.
NTRACKS Number of tracks (inside a jet).
P 		Magnitude of the three-momentum.
PHI 	Azimuthal angle.
PT 		Transverse momentum.
PX 		x-component of the momentum.
PY 		y-component of the momentum.
PZ 		z-component of the momentum.
R 		Position in the (η, φ) plane.
Y 		Rapidity.

DELTAR 		Angular distance, in the transverse plane, between the objects.
DPHI_0_PI 	Angular distance in azimuth between the objects. The bounds for the angle are [0, π].
DPHI_0_2PI 	Angular distance in azimuth between the objects. The bounds for the angle are [0, 2π].

\fi
%%%%%%%%%%%%%%%%%%%%%%%%